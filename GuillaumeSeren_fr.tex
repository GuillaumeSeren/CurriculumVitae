% --------------------------------------------------
% @author Guillaume Seren
% source  https://github.com/GuillaumeSeren/CurriculumVitae
% file    GuillaumeSeren_fr.tex
% Licence GPLv3
%
% Latex main file.
% --------------------------------------------------
%!TEX TS-program = xelatex

\documentclass[10pt,a4paper,sans]{moderncv}

\usepackage{GuillaumeSeren}
\linespread{0.9}
\moderncvstyle{classic} % thème visuel au choix : empty, classic, casual, banking
\moderncvcolor{green}  %couleur du thème au choix : blue, orange, green, red, purple, grey, black
\setlength{\hintscolumnwidth}{3cm} % pour changer la largeur de la colonne de gauche du CV (celle des dates, tout ça)
\usepackage[utf8]{inputenc} % l'encodage que vous utilisez, à adapter à votre cas

\firstname{Guillaume}
\familyname{Seren}
\title{Ingénieur d'Études Freelance}
\address{36, Avenue Victor Cresson}{92130, Issy les Moulineaux}
\email{guillaumeseren@gmail.com}
\extrainfo{https://github.com/GuillaumeSeren}
\homepage{guillaumeseren.com}
\quote{« As long as I'm learning something, I figure I'm OK - it's a decent day. » Hunter S. Thompson}

\hyphenpenalty=10000

\renewcommand{\labelitemi}{-- }
\begin{document}

\makecvtitle

\section{Formation}
\tlcventry{2008}{2009}{Formation POO}{CESI}{Aix-en-Provence}
{Formation J2EE \& DotNet.}{}
\tlcventry{2005}{2007}{BTS Informatique de gestion}{Ecole Charlotte Grawitz}
{Marseille}{Option Développeur d'applications}{}
\tlcventry{2002}{2004}{Deug de psychologie}{Faculté de lettres}{Aix-en-Provence}
{}{}
\tlcventry{2001}{2002}{Baccalauréat STI}{Lycée Antonin Artaud}{Marseille}
{Option Électronique}{}

\section{Compétences}
\subsection{Développement}
\cvlistitem {Coté - serveur: Php, Bash, JavaScript, Java / Jsp, Python.}
\cvlistitem {Coté - client: JavaScript, Bash, Python, Java / J2EE, Ruby.}
\cvlistitem {Présentation: xml, html, xhtml.}
\cvlistitem {Mise en page: CSS}

\subsection{Systèmes d'exploitation}
\cvlistitem {Serveurs: Debian, BSD}
\cvlistitem {Clients: Debian. Osx}
\cvlistitem {Mobile: Android}

\subsection{Administration système}
\cvlistitem {Mise en place d'une architecture de production, avec redondance de serveurs.}
\cvlistitem {Mise en place d'une plateforme de développement collaboratif avec versionning, en utilisant des méthodes agiles / Extreme programming.}
\cvlistitem {Rétroportage de logiciels et correctifs, en utilisant plusieurs branches de maturation.}
\cvlistitem {Mise en place et évolution de serveur WEB de dev / test / préprod, et de production, de type LAMP (Linux Apache Mysql Php).}

\subsection{Méthodes}
\cvlistitem {Diagrammes UML.}
\cvlistitem {Modélisation de donnée MERISE, MCD / MPD.}
\cvlistitem {Suivi de cas d'utilisation et optimisation des fonctionnalité existantes (API).}
\cvlistitem {Utilisation de méthodes agiles.}

\subsection{Réseaux}
\cvlistitem {Gestion de postes client au moyen d'un serveur LDAP / active directory.}
\cvlistitem {Gestion de parc physique et virtuel de machine, screen et scripting de gestion (bash/python).}
\cvlistitem {Réalisation de ghost systèmes, et VM de machine de bench / test.}

\subsection{Sécurité informatique}
\cvlistitem {Audit de serveurs, nmap / metasploit.}
\cvlistitem {Analyse de code, refactoring et optimisations.}
\cvlistitem {Test d'intrusions, mise en place de script d'attaque / bench.}

\section{Expériences}

\tlcventry{2013}{2015}{Ingénieur d'études}{Starpass / BdMultimédia}
{Paris}{}{
    \begin{itemize}%
        \item Mise en place de Méthodes Agile, avec SCRUM.
        \item Développement d'un SCM → GIT, pour gestion de l'application Starpass, sous la forme de 3 branches de codes de bases, sur lesquelles s'appuient les évolutions / modifications, validations, production.
        \item Réalisation de script de snapshot, pour faciliter le suivi et la transition, vers le SCM.
        \item Script de création et mise en place d'utilisateurs Unix, permettant aux différents utilisateurs l'accès à leur espace personnalisé, avec plateforme de test, et accès aux divers scripts.
        \item Script de déploiement, local, et distant, avec un support d'appel de scripts de test, pré/post opératoires.
        \item Script de test fonctionnel, retourne les status HTTP, des ressources et leur temps de réponse, afin de permettre benchmarking du temps et analyse fonctionnel de l'application.
        \item Script d'audit de code PHP, et JS, analyse du code inutile, problèmes d'encodages, path, include, indentation, etc.
        \item Formation aux Méthodes Agiles, de l'équipe en place et rédaction de documentations.
        \item Mise en place de cygwin, sur les machines non-unix, afin de pouvoir utiliser directement les scripts réalisés.
        \item Réalisation d'une étude, avec specifications techniques pour le développement de modules de com et jeu supplémentaires (UML classes, séquences, MCD).
        \item Création d'un MCD pour la DB existante.
        \item Refactoring / versionning de la DB (structure) ainsi que des patch d'évolutions.
        \item Ajout d'un système de gestion des évolution (patch) pour faciliter le suivi des différentes versions du projet.
            \begin{itemize}%
                \item Environnement Tech :
                \item OS : Debian 6 / 7.
                \item Bases de données : Mysql 5.5.
                \item EDI : Vim, Netbeans, Mysql Workbench.
                \item Technologies : PHP, JS, SQL, Py, Git, Bash, git.
            \end{itemize}
\end{itemize}}

\tlcventry{2012}{2013}{Ingénieur d'études}{Renault Retail Group / Purjus}
{Paris}{}{
    \begin{itemize}%
        \item Création d'une plateforme centralisé http://renault-retail-group.fr.
        \item Développement de 7 espaces dédiés régionaux, comme http://paris.renault-retail-group.fr/.
        \item Modélisation et mise en place d'une base de donnée centralisé, en suivant la méthode MERISE.
        \item Mise en place d'une plateforme de développement collaboratif basé sur, svn et trac, pour permettre le suivi et traitement des demandes et modifications.
        \item Développement de diagrammes uml, et implémentation dans le projet.
        \item Mise en place d'une plateforme de pré-production.
        \item Gestion d'un dépôt svn, avec plusieurs branches principales : Production / Pré-production / Développement.
        \item Chaque correctifs, et/ou fonctionnalités sont entreposés dans leur propres branches.
        \item Développement des fonctionnalités en suivant le framework Joomla.
        \item Réalisations de vues mysql.
            \begin{itemize}%
                \item Environnement Tech :
                \item OS : Debian, freeBSD.
                \item Bases de données : Mysql 5.5.
                \item Framework : Joomla.
                \item EDI : Eclipse, Netbeans, vim, mysql workbench.
                \item Librairies : sh404SEF, JCE.
                \item Technologies : PHP, JS, SQL, Py, SVN.
            \end{itemize}
\end{itemize}}

\tlcventry{2010}{2012}{Développeur}{Renault Paris / Purjus}
{Paris}{} {
    \begin{itemize}%
        \item Développements sur le site internet, http://renaultparis.fr.
        \item Développement du module CRUD de gestion des leads et demande client depuis le back-end.
        \item Outil de demande d'informations, et de prise de contacts clients depuis le front-end.
        \item Affichage du stock disponible, et gestion du processus de lead sur un produit.
        \item Mise en place d'une plateforme de versionning, SVN.
            \begin{itemize}%
                \item Environnement Tech :
                \item OS : Debian, freeBSD.
                \item Bases de données : Mysql 5.5.
                \item Framework : Joomla.
                \item EDI : Eclipse, Netbeans, vim.
                \item Librairies : sh404SEF, JCE.
                \item Technologies : PHP, JS, SQL, Py, SVN.
            \end{itemize}
\end{itemize}}

\tlcventry {2008}{2010} {Ingénieur d'études} {KP1 / GFI}
{Avignon}{} {
    \begin{itemize}%
        \item Réalisation d'un moteur de calcul basé sur des formules.
        \item Stockage d'une partie des paramètres dans une base de donnée.
        \item Mise un place d'un cache objet Jcs.
        \item Réalisation d'une batterie de test unitaire basé sur jUnit.
            \begin{itemize}%
                \item Environnement Tech :
                \item OS : Windows Xp, Windows 2003 Serveur, VMWare.
                \item Bases de données : SQL Serveur 2005
                \item Framework : JDK 1.5,  Struts 1.x.
                \item EDI : Eclipse 3.4.1 Ganymede, NetBeans 6.1
                \item Librairies : Tiles, Validator, Jep, Jcs, jUnit.
                \item Technologies : JAVA, Java Script, CSS, JSP, CVS.
            \end{itemize}
\end{itemize}}

\tlcventry {2007}{2008} {Administrateur systèmes} {Mairie de Marseille / Computacenter}
{Marseille}{} {
    \begin{itemize}%
        \item Administration de serveurs pré-production et production.
        \item Mise en place de test de redémarrage des services.
        \item Mise en place de test de montée en charge.
        \item Sauvegarde des données sensibles.
        \item Travail en environnement hétérogène.
            \begin{itemize}%
                \item Environnement Tech :
                \item OS : WinXp, Win2003, Debian, RedHat Entreprise, Ubuntu.
                \item Bases de données : Oracle 9.x, MySQL 4.x, Sql Serveur 2005.
                \item EDI : NotePad++, vi.
                \item Technologies : SSH, Bash linux et windows.
            \end{itemize}
\end{itemize}}

\tlcventry {2005}{2007} {Développeur} {Liligo / FINDWORKS TECHNOLOGIES}
{Paris}{} {
    \begin{itemize}%
        \item Travail sur le projet : http://www.liligo.com/
        \item Gestion de bases de données.
        \item Développement de script de parsing coté serveur en JS.
        \item Développement de script de parsing de site web, en PERL.
        \item Développement d'IHM de l'application Web.
        \item Développement du modèle XHTML suivant une charte graphique.
        \item Développement de l’habillage du style de l'application en CSS.
        \item Analyse et gestion du Search Engine Optimisation, SEO.
        \item Contrôle qualité et test de montée en charge.
            \begin{itemize}%
                \item Environnement Tech :
                \item OS : Debian, WinXP.
                \item Bases de données : MySQL 4,x
                \item Framework : JDK 1,4.
                \item EDI : Eclipse 3,2 Web Tools Platform.
                \item librairies : hibernate.
                \item Technologies : SSH, JAVA, JSP, Java Script, CSS, xHTML, PERL.
            \end{itemize}
\end{itemize}}

\clearpage
\end{document}
