\begin{document}
\header{Guillaume}{Seren}
      {Ingénieur d'Études Freelance}

\begin{aside}
  \section{Address}
    36, Avenue Victor Cresson
    92130, Issy les Moulineaux
    ~
  \section{Mail}
    \href{mailto:guillaumeseren@gmail.com}{\textbf{guillaumeseren@gmail.com}}
    ~
  \section{Web \& Git}
    \href{http://guillaumeseren.com/}{guillaumeseren.com}
    \href{https://github.com/GuillaumeSeren}{github.com/GuillaumeSeren}
    ~
\end{aside}

\section{Education}
\begin{entrylist}
  \entry
    {2008}
    {Formation POO}
    {CESI, Aix-en-Provence}
    {Formation J2EE \& DotNet.}
  \entry
    {2007}
    {BTS Informatique de gestion}
    {Ecole Charlotte Grawitz, Marseille}
    {Option « Développeur d'applications ».}
  \entry
    {2004}
    {Deug de psychologie}
    {Faculté de lettres, Aix-en-Provence}
    {}
  \entry
    {2002}
    {Baccalauréat STI}
    {Lycée Antonin Artaud, Marseille}
    {Option « Électronique »}
\end{entrylist}

\section{Compétences}
\begin{entrylist}
  \entry
  {Développement :}
  {Coté - serveur : Php, Bash, JavaScript, Java / Jsp, Python, C# / Asp.
Coté - client : JavaScript, Bash, Python, Java / J2EE, C# / .Net.
Présentation : xml, html, xhtml.
Mise en page : CSSx / CSS3.}

  \entry
  {Systèmes :}
  {Serveurs : Debian, Gentoo, BSD, Slackware, NixOs.
  Clients : Debian. Android, Osx, Windows, FirefoxOs, TizenOs.}

  \entry
  {Administration système :}
  {Mise en place d'une architecture de production, avec redondance de serveurs.
Mise en place d'une plateforme de développement collaboratif avec versionning, en utilisant des méthodes agiles / Extreme programming.
Rétroportage de logiciels et correctifs, en utilisant plusieurs branches de maturation.
Mise en place et évolution de serveur WEB de dev / test / préprod, et de production, de type LAMP (Linux Apache Mysql Php).}

  \entry
  {Modélisation :}
  {Diagrammes UML.
Modélisation de donnée MERISE, MCD / MPD.
Suivi de cas d'utilisation et optimisation des fonctionnalité existantes (API).}

  \entry
  {Réseaux :}
  {Gestion de postes client au moyen d'un serveur LDAP / active directory.
Gestion de parc physique et virtuel de machine, screen et scripting de gestion (bash/python).
Réalisation de ghost systèmes, et VM de machine de bench / test.}

  \entry
  {Sécurité informatique :}
  {Audit de serveurs, nmap / metasploit.
Analyse de code, refactoring et optimisations.
Test d'intrusions, mise en place de script d'attaque / bench.}
\end{entrylist}


\section{Expériences}
\begin{entrylist}
  \entry
  {Starpass / BdMultimédia}
  {Paris (Mar 2013 – 2015)}
  {Poste : Ingénieur d'études.}
  {Mission :
- Mise en place de Méthodes Agile, avec « SCRUM ».
- Développement d'un SCM → GIT, pour gestion de l'application Starpass, sous la forme de 3 branches de codes de bases, sur lesquelles s'appuient les évolutions / modifications, validations, production.
- Réalisation de script de snapshot, pour faciliter le suivi et la transition, vers le SCM.
- Script de création et mise en place d'utilisateurs « Unix », permettant aux différents utilisateurs l'accès à leur espace personnalisé, avec plateforme de test, et accès aux divers scripts.
- Script de déploiement, local, et distant, avec un support d'appel de scripts de test, pré/post opératoires.
- Script de test fonctionnel, retourne les status « HTTP », des ressources et leur temps de réponse, afin de permettre benchmarking du temps et analyse fonctionnel de l'application.
- Script d'audit de code PHP, et JS, analyse du code inutile, problèmes d'encodages, path, include, indentation, etc.
- Formation aux Méthodes Agiles, de l'équipe en place et rédaction de documentations.
- Mise en place de cygwin, sur les machines non-unix, afin de pouvoir utiliser directement les scripts réalisés.
- Réalisation d'une étude, avec specifications techniques pour le développement de modules de com et jeu supplémentaires (UML classes, séquences, MCD).
- Création d'un MCD pour la DB existante.
- Refactoring / versionning de la DB (structure) ainsi que des patch d'évolutions.
- Ajout d'un système de gestion des évolution (patch) pour faciliter le suivi des différentes versions du projet.
Environnement Tech :
- OS : Debian 6 / 7.
- Bases de données : Mysql 5.5.
- EDI : Vim, Netbeans, Mysql Workbench.
- Technologies : PHP, JS, SQL, Py, Git, Bash, git.}

  \entry
  {Renault Retail Group}
  {Paris (jan 2012 – jan 2013)}
  {Poste : Ingénieur d'études.}
  {Mission :
- Création d'une plateforme centralisé « http://renault-retail-group.fr ».
- Développement de 7 espaces dédiés régionaux, comme « http://paris.renault-retail-group.fr/ ».
- Modélisation et mise en place d'une base de donnée centralisé, en suivant la méthode MERISE.
- Mise en place d'une plateforme de développement collaboratif basé sur, svn et trac, pour permettre le suivi et traitement des demandes et modifications.
- Développement de diagrammes uml, et implémentation dans le projet.
- Mise en place d'une plateforme de pré-production.
- Gestion d'un dépôt svn, avec plusieurs branches principales : Production / Pré-production / Développement.
- Chaque correctifs, et/ou fonctionnalités sont entreposés dans leur propres branches.
- Développement des fonctionnalités en suivant le framework Joomla.
- Réalisations de vues mysql.
Environnement Tech :
- OS : Debian, freeBSD.
- Bases de données : Mysql 5.5.
- Framework : Joomla.
- EDI : Eclipse, Netbeans, vim, mysql workbench.
- Librairies : sh404SEF, JCE.
- Technologies : PHP, JS, SQL, Py, SVN.}

  \entry
  {Renault Paris}
  {Paris (mai 2010 – jan 2012)}
  {Poste : Développeur.}
  {Mission :
- Développements sur le site internet, « http://renaultparis.fr ».
- Développement du module CRUD de gestion des leads et demande client depuis le back-end.
- Outil de demande d'informations, et de prise de contacts clients depuis le front-end.
- Affichage du stock disponible, et gestion du processus de lead sur un produit.
- Mise en place d'une plateforme de versionning, SVN.
Environnement Tech :
- OS : Debian, freeBSD.
- Bases de données : Mysql 5.5.
- Framework : Joomla.
- EDI : Eclipse, Netbeans, vim.
- Librairies : sh404SEF, JCE.
- Technologies : PHP, JS, SQL, Py, SVN.}

  \entry
  {KP1 – GFI}
  {Avignon (aout 2008 – jan 2010)}
  {Poste : Ingénieur d'études}
  {Mission :
- Réalisation d'un moteur de calcul basé sur des formules.
- Stockage d'une partie des paramètres dans une base de donnée.
- Mise un place d'un cache objet Jcs.
- Réalisation d'une batterie de test unitaire basé sur jUnit.
Environnement Tech :
- OS : Windows Xp, Windows 2003 Serveur, VMWare.
- Bases de données : SQL Serveur 2005
- Framework : JDK 1.5,  Struts 1.x.
- EDI : Eclipse 3.4.1 Ganymede, NetBeans 6.1
- Librairies : Tiles, Validator, Jep, Jcs, jUnit.
- Technologies : JAVA, Java Script, CSS, JSP, CVS.}

  \entry
  {Mairie de Marseille}
  {Marseille (Jan 2008 – mars 2008)}
  {Poste : Administrateur systèmes}
  {Mission :
- Administration de serveurs pré-production et production.
- Mise en place de test de redémarrage des services.
- Mise en place de test de montée en charge.
- Sauvegarde des données sensibles.
- Travail en environnement hétérogène.
Environnement Tech :
- OS : WinXp, Win2003, Debian, RedHat Entreprise, Ubuntu.
- Bases de données : Oracle 9.x, MySQL 4.x, Sql Serveur 2005.
- EDI : NotePad++, vi.
- Technologies : SSH, Bash linux et windows.}

  \entry
  {Liligo - FINDWORKS TECHNOLOGIES}
  {Paris (2005 - 2007)}
  {Poste : Développeur}
  {Mission :
- Travail sur le projet : « http://www.liligo.com/ »
- Gestion de bases de données.
- Développement de script de parsing coté serveur en JS.
- Développement de script de parsing de site web, en PERL.
- Développement d'IHM de l'application Web.
- Développement du modèle XHTML suivant une charte graphique.
- Développement de l’habillage du style de l'application en CSS.
- Analyse et gestion du Search Engine Optimisation, SEO.
- Contrôle qualité et test de montée en charge.
Environnement Tech :
- OS : Debian, WinXP.
- Bases de données : MySQL 4,x
- Framework : JDK 1,4.
- EDI : Eclipse 3,2 Web Tools Platform.
- librairies : hibernate.
- Technologies : SSH, JAVA, JSP, Java Script, CSS, xHTML, PERL.}
\end{entrylist}

\section{
\begin{flushleft}
\emph{Mercredi 10 Juin 2015}
\end{flushleft}

\begin{flushright}
\emph{Guillaume Seren}
\end{flushright}

\end{document}
